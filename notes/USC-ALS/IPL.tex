% arara: pdflatex
% arara: bibtex
% arara: pdflatex
% arara: pdflatex
\documentclass[12pt]{article}

\usepackage{lmodern}
\usepackage{microtype}

% Package for customizing page layout
%\usepackage[letterpaper]{geometry}
\usepackage[
top    = 3cm,
bottom = 3cm,
left   = 4.00cm,
right  = 4.00cm]{geometry}
%\usepackage[letterpaper]{geometry}
% \usepackage{fullpage}

%%% This file can never be completed.
%%% If you need something but cannot find it,
%%% contact the TA Favonia!

%%%%%%%%%%%%%%%%%%%%%%%%%%%%%%%%%%%%%%%%
% Basic packages
%%%%%%%%%%%%%%%%%%%%%%%%%%%%%%%%%%%%%%%%
\usepackage{amsmath,amsthm,amssymb}
\usepackage{mathtools}
\usepackage{etoolbox}
\usepackage{fancyhdr}
\usepackage{mathpartir}
\usepackage{xcolor}
\usepackage{hyperref}
\usepackage{xspace}
\usepackage{comment}
\usepackage{url} % for url in bib entries

%%%%%%%%%%%%%%%%%%%%%%%%%%%%%%%%%%%%%%%%
% Acronyms
%%%%%%%%%%%%%%%%%%%%%%%%%%%%%%%%%%%%%%%%
\usepackage[acronym, shortcuts]{glossaries}

\newacronym{HoTT}{HoTT}{homotopy type theory}
\newacronym{IPL}{IPL}{intuitionistic propositional logic}
\newacronym{TT}{TT}{intuitionistic type theory}
\newacronym{LEM}{LEM}{law of the excluded middle}
\newacronym{ITT}{ITT}{intensional type theory}
\newacronym{ETT}{ETT}{extensional type theory}
\newacronym{NNO}{NNO}{natural numbers object}

% Make \ac robust.
\robustify{\ac}

%%%%%%%%%%%%%%%%%%%%%%%%%%%%%%%%%%%%%%%%
% Fancy page style
%%%%%%%%%%%%%%%%%%%%%%%%%%%%%%%%%%%%%%%%
\pagestyle{fancy}
\newcommand{\metadata}[2]{
  \lhead{Algebra and Logic Seminar}
  \chead{IPL Notes}
  \rhead{February 28, 2014}
  %% \lfoot{#1}
  %% \cfoot{#2}
  %% \rfoot{\thepage}
  \lfoot{}
  \cfoot{}
  \rfoot{}
}
\renewcommand{\headrulewidth}{0.4pt}
%\renewcommand{\footrulewidth}{0.4pt}


\newrobustcmd*{\vocab}[1]{\emph{#1}}
\newrobustcmd*{\latin}[1]{\textit{#1}}

%%%%%%%%%%%%%%%%%%%%%%%%%%%%%%%%%%%%%%%%
% Customize list enviroonments
%%%%%%%%%%%%%%%%%%%%%%%%%%%%%%%%%%%%%%%%
% package to customize three basic list environments: enumerate, itemize and description.
\usepackage{enumitem}
\setitemize{noitemsep, topsep=0pt, leftmargin=*}
\setenumerate{noitemsep, topsep=0pt, leftmargin=*}
\setdescription{noitemsep, topsep=0pt, leftmargin=*}

%%%%%%%%%%%%%%%%%%%%%%%%%%%%%%%%%%%%%%%%
% Some really basic macros.
% (Lots of them were stolen from HoTT/Book.)
% See macros.tex in HoTT/book.
%
% This is a mess.  Needs clean-ups.
%%%%%%%%%%%%%%%%%%%%%%%%%%%%%%%%%%%%%%%%
\newrobustcmd*{\ctx}{\Gamma}
\newrobustcmd*{\entails}{\vdash}

\newrobustcmd*{\judgmentfont}[1]{{\normalfont\sffamily #1}}
\newrobustcmd*{\postfixjudgment}[1]{%
  \relax\ifnum\lastnodetype>0\mskip\medmuskip\fi
  \text{\judgmentfont{#1}}%
}
\newrobustcmd*{\prop}{\postfixjudgment{prop}}
\newrobustcmd*{\true}{\postfixjudgment{true}}
\newrobustcmd*{\type}{\postfixjudgment{type}}
\newrobustcmd*{\context}{\postfixjudgment{ctx}}

\newrobustcmd*{\truth}{\top}
\newrobustcmd*{\conj}{\wedge}
\newrobustcmd*{\disj}{\vee}
\newrobustcmd*{\falsehood}{\bot}
\newrobustcmd*{\imp}{\supset}
\newrobustcmd*{\zero}{0}

%%% Judgmental equality
\newrobustcmd*{\jdeq}{\equiv}
%%% Definition
\newrobustcmd*{\defeq}{\vcentcolon\equiv}
%%% Binary sums
\newrobustcmd*{\inlsym}{{\mathsf{inl}}}
\newrobustcmd*{\inrsym}{{\mathsf{inr}}}
\newrobustcmd*{\inl}{\ensuremath\inlsym\xspace}
\newrobustcmd*{\inr}{\ensuremath\inrsym\xspace}
%%% Booleans
\newrobustcmd*{\ttsym}{{\mathsf{tt}}}
\newrobustcmd*{\ffsym}{{\mathsf{ff}}}
%%% Pairs
\newrobustcmd*{\pair}{\ensuremath{\mathsf{pair}}\xspace}
\newrobustcmd*{\tuple}[2]{(#1,#2)}
\newrobustcmd*{\proj}[1]{\ensuremath{\mathsf{pr}_{#1}}\xspace}
%% Empty type
\newrobustcmd*{\abort}[1]{\ensuremath{\mathsf{abort}_{#1}}}
%%% Path concatenation
\newrobustcmd*{\concat}{%
  \mathchoice{\mathbin{\raisebox{0.5ex}{$\displaystyle\centerdot$}}}%
  {\mathbin{\raisebox{0.5ex}{$\centerdot$}}}%
  {\mathbin{\raisebox{0.25ex}{$\scriptstyle\,\centerdot\,$}}}%
  {\mathbin{\raisebox{0.1ex}{$\scriptscriptstyle\,\centerdot\,$}}}
}
%%% Transport (covariant)
\newrobustcmd*{\trans}[2]{\ensuremath{{#1}_{*}\mathopen{}\left({#2}\right)\mathclose{}}\xspace}
% Natural numbers objects
\newrobustcmd*{\Nat}{\mathsf{Nat}}
\newrobustcmd*{\rec}{\ensuremath{\mathsf{rec}}\xspace}
% Sequence
\newrobustcmd*{\Seq}{\ensuremath{\mathsf{Seq}}\xspace}
% Identity type
\newrobustcmd*{\Id}[1]{\ensuremath{\mathsf{Id}_{#1}}\xspace}
% Reflection
\newrobustcmd*{\refl}[1]{\ensuremath{\mathsf{refl}_{#1}}\xspace}
\newrobustcmd*{\J}{\ensuremath{\mathsf{J}}\xspace}

% fst,snd,case,id
\newrobustcmd*{\fst}{\textsf{fst}}
\newrobustcmd*{\snd}{\textsf{snd}}
\DeclareMathOperator{\case}{\textsf{case}}
\DeclareMathOperator{\caseif}{\textsf{if}}
\DeclareMathOperator{\casesplit}{\textsf{split}}
\DeclareMathOperator{\ttrue}{\textsf{tt}\xspace}
\DeclareMathOperator{\ffalse}{\textsf{ff}\xspace}
\newrobustcmd*{\id}{\textsf{id}}

\newrobustcmd*{\op}[1]{\operatorname{#1}}

\newrobustcmd*{\universe}{\mathcal{U}}

%inductive types
\newrobustcmd*{\ind}{\ensuremath{\mathsf{ind}}\xspace}
%higher inductive types
%interval
\newrobustcmd*{\interval}{\ensuremath{I}\xspace}
\newrobustcmd*{\seg}{\ensuremath{\mathsf{seg}}\xspace}
%circle
\newrobustcmd*{\Sn}{\mathbb{S}}
\newrobustcmd*{\base}{\ensuremath{\mathsf{base}}\xspace}
\newrobustcmd*{\lloop}{\ensuremath{\mathsf{loop}}\xspace}


\usepackage{proof-dashed}
\usepackage{tikz-cd}

\metadata{IPL Notes}{Feb 28, 2014}


\begin{document}

%% \title{Rules of Inference in IPL}
%% %\author{}
%% \date{February 21 and 28, 2014}

%\maketitle

\section{Introduction}

This is a summary of the rules of inference for intuitionistic propositional
logic as I understand them.  This is based on the lectures given by Professor
Robert Harper in September 2013, at CMU. Notes for Harper's lectures were transcribed
by his students and this summary is a highly abridged and edited version of the
students' notes. 

As advanced by Per Martin-L\"{o}f, a modern presentation of \acf{IPL}
distinguishes the notions of \vocab{judgment} and \vocab{proposition}. A
judgment is something that may be known, whereas a proposition is something that
sensibly may be the subject of a judgment. For instance, the statement ``Every
natural number larger than $1$ is either prime or can be uniquely factored into
a product of primes\@.'' is a proposition because it sensibly may be subject to
judgment. That the statement is in fact true is a judgment.
Only with a proof, however, is it evident that the judgment indeed holds.

Thus, in \ac{IPL}, the two most basic judgments are $A \prop$ and $A \true$:
\begin{alignat*}{2}
  A \prop &&\quad& \text{$A$ is a well-formed proposition} \\
  A \true &&& \text{\begin{tabular}[t]{@{}l@{}}
                Proposition $A$ is intuitionistically true, \\
                i.e., has a proof.
              \end{tabular}}
\end{alignat*}
The inference rules for the $\prop$ judgment are called formation rules.
The inference rules for the $\true$ judgment are divided into classes:
\vocab{introduction rules} and \vocab{elimination rules}. 

Following Martin-L\"{o}f, the meaning of a proposition $A$ is given by the
introduction rules for the judgment $A \true$. The elimination rules are dual
and  describe what may be deduced from a proof of $A \true$. 

The principle of \vocab{internal coherence}, also known as Gentzen's principle
of inversion, is that the introduction and elimination rules for a proposition
$A$ fit together properly.  The elimination rules should be strong enough to
deduce all information that was used to introduce $A$ (\vocab{local
  completeness}), but not so strong as to deduce information that might not have
been used to introduce $A$ (\vocab{local soundness}).  In a later lecture, we
will discuss internal coherence more precisely, but we can already give an
informal treatment. 

\newpage

\section{The negative fragment of IPL}

\paragraph{Conjunction.}
%\subsection{Conjunction}
\begin{itemize}
\item[Formation:] 
If $A$ and $B$ are well-formed propositions, then so is
their \emph{conjunction}, which we write as $A \conj B$.

\begin{equation*}
  \infer[{\conj}\mathsf{F}]{A \conj B \prop}{
    A \prop & B \prop}
\end{equation*}

\item[Introduction:]
%\paragraph{Introduction.}
To give meaning to conjunction, we must say how
to introduce the judgment $A \conj B \true$.
A verification of $A \conj B$ requires a proof of $A$ and
a proof of $B$.

\begin{equation*}
  \infer[{\conj}\mathsf{I}]{A \conj B \true}{
    A \true & B \true}
\end{equation*}

\item[Elimination:]
%\paragraph{Elimination.}
Because every proof of $A \conj B$ comes from a pair of proofs, one of $A$ and
one of $B$, we are justified in deducing $A \true$ and $B \true$ from a proof of $A \conj B$:
\begin{mathpar}
  \infer[{\conj}\mathsf{E}_1]{A \true}{
    A \conj B \true}
  \and
  \infer[{\conj}\mathsf{E}_2]{B \true}{
    A \conj B \true}
\end{mathpar}

\end{itemize}

\subsection{Truth}
\paragraph{Formation.} 
Another proposition is \vocab{truth}, which we denote $\truth$.  It is 
the \emph{trivially true} proposition, and its formation rule serves as
immediate evidence for the judgment that $\truth$ is indeed a
well-formed proposition.
\begin{equation*}
  \infer[{\truth}\mathsf{F}]{\truth \prop}{
    }
\end{equation*}

\paragraph{Introduction.}
To give meaning to truth we say how to introduce the judgment that $\truth$ is true.
Since $\truth$ is a trivially true proposition, its introduction rule makes the
judgment $\truth \true$ immediately evident.
\begin{equation*}
  \infer[{\truth}\mathsf{I}]{\truth \true}{
    }
\end{equation*}

\paragraph{Elimination.}
Since $\truth$ is trivially true, an elimination rule should not increase
our knowledge---we put in no information when we introduced $\truth \true$, so,
by the principle of conservation of proof, we should get no information out. For
this reason, there is no elimination rule for $\truth$.

\subsection{Entailment}
\emph{Entailment} is a judgment. It is written as 
\begin{equation*}
  A_1 \true, \dotsc, A_n \true \entails A \true
\end{equation*}
and expresses the judgment that $A \true$ follows from $A_1 \true, \dotsc, A_n \true$.
One can view $A_1 \true, \dotsc, A_n \true$ as being assumptions from which
the conclusion $A \true$ may be deduced. 

%% The metavariable $\ctx$ is typically
%% used to stand for such a context of assumptions. 

%% We should note that, thus far, the inference rules have been presented in a
%% \vocab{local form} in which the context of assumptions was left implicit. 
%% It would also be possible to make this context explicit.
%% For example, the introduction rule for conjunction could be written in context as
%% \begin{mathpar}
%%   \infer[{\conj}I]{\ctx \entails A \conj B \true}{
%%     \ctx \entails A \true &
%%     \ctx \entails B \true}
%% \end{mathpar}

We assume that the entailment judgment satisfies several \emph{structural
  properties}: reflexivity, transitivity, weakening, contraction, and
permutation. 

\paragraph{Reflexivity.}
(An assumption is enough to conclude the same judgment.)
\begin{equation*}
  \infer[\text{\textsf{R}}]{A \true \entails A \true}{
    }
\end{equation*}


\paragraph{Transitivity.}
(If you prove $A \true$, then you are justified in using it in a proof.)
\begin{equation*}
  \infer[\text{\textsf{T}}]{C \true}{
    A \true &
    A \true \entails C \true}
\end{equation*}


Reflexivity and transitivity are undeniable since assumptions should be strong enough
to prove conclusions (reflexivity), and only as strong as the proofs they stand for
(transitivity). 
The remaining structural properties---weakening, contraction, and
permutation---could be denied.  Logics that deny any of these properties are
called \emph{substructural logics}. 

\paragraph{Weakening.}
We can add assumptions to a proof without invalidating that proof.
\begin{equation*}
  \infer[\text{\textsf{W}}]{B \true \entails A \true}{
    A \true}
\end{equation*}

\paragraph{Contraction.}
The number of copies of an assumption does not matter.
\begin{equation*}
  \infer[\text{\textsf{C}}]{A \true \entails C \true}{
    A \true, A \true \entails C \true}
\end{equation*}

\paragraph{Permutation.}
aka ``exchange;'' the order of assumptions does not matter.
\begin{equation*}
  \infer[\text{\textsf{P}}]{\pi(\ctx) \entails C \true}{
    \ctx \entails C \true}
\end{equation*}

\subsection{Implication}

\paragraph{Formation.} 
\begin{equation*}
  \infer[{\imp}F]{A \imp B \prop}{
    A \prop & B \prop}
\end{equation*}

\paragraph{Introduction.}
\begin{equation*}
  \infer[{\imp}I]{A \imp B \true}{
    A \true \entails B \true}
\end{equation*}
In this way, implication internalizes the entailment judgment as a proposition,
while we nonetheless maintain the distinction between propositions and
judgments.
%% (As an aside for those familiar with category theory, the relationship between
%% entailment and implication is analogous to the relationship between a mapping
%% and a collection of mappings internalized as an object in some category.) 

\paragraph{Elimination.}
\begin{equation*}
  \infer[{\imp}E]{B \true}{
    A \imp B \true & A \true} \,.
\end{equation*}
This rule is sometimes referred to as \latin{modus ponens}.


\section{The positive fragment of \ac{IPL}}

\subsection{Disjunction}
\paragraph{Formation.} 
\begin{equation*}
  \infer[{\disj}F]{A \disj B \prop}{
    A \prop & B \prop}
\end{equation*}

\paragraph{Introduction.}
\begin{mathpar}
  \infer[{\disj}I_1]{A \disj B \true}{
    A \true}
  \and
  \infer[{\disj}I_2]{A \disj B \true}{
    B \true}
\end{mathpar}

\paragraph{Elimination.}
\begin{equation*}
  \infer[{\disj}E]{C \true}{
    A \disj B \true &
    A \true \entails C \true & B \true \entails C \true}
\end{equation*}

\subsection{Falsehood}\label{sec:falsehood}
\paragraph{Formation.} 
The unit of disjunction is falsehood, the proposition that is trivially never
true, which we write as $\falsehood$.  Its formation rule is immediate evidence
that $\falsehood$ is a well-formed proposition. 
\begin{equation*}
  \infer[{\falsehood}F]{\falsehood \prop}{
    }
\end{equation*}

\paragraph{Introduction.}
Because $\falsehood$ should never be true, it has no introduction rule.

\paragraph{Elimination.}
\begin{equation*}
  \infer[{\falsehood}E]{C \true}{
    \falsehood \true}
\end{equation*}
The elimination rule captures \latin{ex falso quodlibet}: from a proof of $\falsehood \true$, we may deduce that \emph{any} proposition $C$ is true because there is ultimately no way to introduce $\falsehood \true$.
Once again, the rules cohere.
The elimination rule is very strong, but remains justified due to the absence of any introduction rule for falsehood.


\section{Order-theoretic formulation of \acs{IPL}}\label{sec:ipl_order}

It is also possible to give an order-theoretic formulation of \ac{IPL} because
entailment is a preorder (reflexive and transitive). We want $A \leq B$ to hold
exactly when $A \true \entails B \true$.  We can therefore devise the
order-theoretic formulation with these soundness and completeness goals in
mind. 

\subsection{Conjunction as meet}\label{sec:conjunction-as-meet}

The elimination rules for conjunction (along with reflexivity of entailment) ensure that $A \conj B \true \entails A \true$ and $A \conj B \true \entails B \true$.
To ensure completeness of the order-theoretic formulation, we include the rules
\begin{mathpar}
  \infer{A \conj B \leq A}{
    }
  \and
  \infer{A \conj B \leq B}{
    } \,,
\end{mathpar}
which say that $A \conj B$ is a lower bound of $A$ and $B$.

The introduction rule for conjunction ensures that $C \true \entails A \conj B \true$ if both $C \true \entails A \true$ and $C \true \entails B \true$.
Order-theoretically, this is expressed as the rule
\begin{equation*}
  \infer{C \leq A \conj B}{
    C \leq A & C \leq B} \,,
\end{equation*}
which says that $A \conj B$ is as large as any lower bound of $A$ and $B$.
Taken together these rules show that $A \conj B$ is the greatest lower bound, or meet, of $A$ and $B$.

Graphically, these order-theoretic rules can be represented with a commuting \vocab{product diagram}, where arrows point from smaller to larger elements:
\begin{equation*}
  \begin{tikzcd}
    {} &[-4ex] C \ar[bend right]{ddl}\ar[bend left]{ddr}\dar[dashed] &[-4ex] {} \\
    {} & A \conj B \dlar\drar & {} \\
    A & {} & B
  \end{tikzcd}
\end{equation*}

\subsection{Truth as greatest element}\label{sec:truth-as-greatest}

The introduction rule for $\truth$ ensures that $C \true \entails \truth \true$.
Order-theoretically, we have 
\begin{equation*}
  \infer{C \leq \truth}{
    } \,,
\end{equation*}
which says that $\truth$ is the greatest, or final, element.

In the proof-theoretic formulation of \ac{IPL}, we saw that truth $\truth$ is the nullary conjunction.
We should expect this analogy to hold in the order-theoretic formulation of \ac{IPL} as well, and it does---the greatest element is indeed the greatest lower bound of the empty set.

\subsection{Disjunction as join}\label{sec:disjunction-as-join}

The introduction rules for disjunction (along with reflexivity of entailment) ensure that $A \true \entails A \disj B \true$ and $B \true \entails A \disj B \true$.
To ensure completeness of the order-theoretic formulation, we include the rules
\begin{mathpar}
  \infer{A \leq A \disj B}{
    }
  \and
  \infer{B \leq A \disj B}{
    } \,,
\end{mathpar}
which say that $A \disj B$ is an upper bound of $A$ and $B$.

The elimination rule for disjunction (along with reflexivity of entailment) ensures that $A \disj B \true \entails C \true$ if both $A \true \entails C \true$ and $B \true \entails C \true$.
Order-theoretically, we have the corresponding rule
\begin{equation*}
  \infer{A \disj B \leq C}{
    A \leq C & B \leq C} \,,
\end{equation*}
which says that $A \disj B$ is as small as any upper bound of $A$ and $B$.
Taken together these rules show that $A \disj B$ is the least upper bound, or join, of $A$ and $B$.

Graphically, this is captured by a commuting \vocab{coproduct diagram}:
\begin{equation*}
  \begin{tikzcd}
    A \drar\ar[bend right]{ddr} &[-4ex] {} &[-4ex] B \dlar\ar[bend left]{ddl}\\
    {} & A \disj B \dar[dashed] & {} \\
    {} & C & {}
  \end{tikzcd}
\end{equation*}

\subsection{Falsehood as least element}\label{sec:falsehood-as-least}

The elimination rule for falsehood (along with reflexivity of entailment) ensures that $\falsehood \true \entails C \true$.
The order-theoretic counterpart is the rule
\begin{equation*}
  \infer{\falsehood \leq C}{
    } \,,
\end{equation*}
which says that $\falsehood$ is the least, or initial, element.

Once again, because we saw that falsehood is the nullary disjunction in the proof-theoretic formulation, we should expect this analogy to carry over to the order-theoretic formulation.
Indeed, the least element is the least upper bound of the empty set.

\subsection{Order-theoretic \ac{IPL} as lattice}\label{subsec:lattice}

As seen thus far, the order-theoretic formulation of \ac{IPL} gives rise to a \vocab{lattice} as it establishes a
preorder with finite meets and joins.  The definition of a lattice assumed in this course may
deviate from the one typically found in the literature, which usually considers a lattice to be
a partial order with finite meets and joins.  In this course, we deliberately ignore the
property of antisymmetry.  If we were to impose the property of antisymmetry on the order
defined by entailment, then we would need to introduce equivalence classes of propositions,
which requires associativity.  As we will see later in this course, the axiom of univalence
provides an elegant way of dealing with equivalence of propositions.

\subsection{Implication as exponential}\label{sec:impl-as-expon}

The elimination rule for implication (along with reflexivity of entailment) ensures that $A \true, A \imp B \true \entails B \true$.
For the order-theoretic formulation to be complete, we include the rule 
\begin{equation*}
  \infer{A \conj (A \imp B) \leq B}{
    }
\end{equation*}
% which say that $A \conj B$ is a lower bound of $A$ and $B$.

The introduction rule for implication ensures that $C \true \entails A \imp B \true$ if $A \true, C \true \entails B \true$.
Once again, so that the order-theoretic formulation is complete, we have
\begin{equation*}
  \infer{C \leq A \imp B}{
    A \conj C \leq B} \,,
\end{equation*}
Taken together, these rules show that $A \imp B$ is the exponential of $A$ and $B$.

% The \vocab{exponential diagram} is:
% \begin{equation*}
%   \begin{tikzcd}
%     C \dar[dashed] \\
%     A \imp B
%   \end{tikzcd}
%   \begin{tikzcd}
%     A \conj (A \imp B) \\
%   \end{tikzcd}
% \end{equation*}

As we have seen previously, the order-theoretic formulation of \ac{IPL} gives
rise to a lattice.  Now we have just seen that it also supports exponentials.
As a result, the order-theoretic formulation of \ac{IPL} gives rise to a
\vocab{Heyting algebra}.  A Heyting algebra is a lattice with exponentials.  As
we will see later in this course, the notion of a Heyting algebra is fundamental
in proving completeness of \ac{IPL}.  The proof also relies on the notion of a
\vocab{complement} in a lattice.  The complement $\overline{A}$ of $A$ in a
lattice is such that
\begin{enumerate}
\item $\top\leq \overline{A}\vee A$;
\item $\overline{A}\wedge A \leq\bot$.
\end{enumerate}
It follows that a complement, if present, is a suitable notion of negation, but
negation, defined via the exponential, is not necessarily a complement.



\nocite{Pfenning2009a, Pfenning2009b}
\bibliographystyle{plain}
\bibliography{hott_references}

\end{document}
